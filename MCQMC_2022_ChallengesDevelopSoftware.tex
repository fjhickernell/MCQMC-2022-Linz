%Talk given at MCQMC 2022 July
\documentclass[11pt,compress,xcolor={usenames,dvipsnames},aspectratio=169]{beamer}
%\documentclass[xcolor={usenames,dvipsnames},aspectratio=169]{beamer} %slides and 
%notes
\usepackage{amsmath,
	amssymb,
	datetime,
	mathtools,
	bbm,
	%mathabx,
	array,
	booktabs,
	xspace,
	multirow,
	calc,
	colortbl,
	siunitx,
 	graphicx}
\usepackage[usenames]{xcolor}
\usepackage[giveninits=false,
    backend=biber,
    giveninits=true,
    style=nature,
    style=authoryear,
    maxcitenames=2,
    mincitenames=1]{biblatex}
\AtBeginBibliography{\small}
\addbibresource{FJHown23.bib}
\addbibresource{FJH23.bib}
\usepackage{newpxtext}
\usepackage[euler-digits,euler-hat-accent]{eulervm}
\usepackage{media9}
\usepackage[autolinebreaks]{mcode}
\usepackage[tikz]{mdframed}

\usepackage[T1]{fontenc}
\usepackage{tgadventor} %Font found at https://tug.org/FontCatalogue/
%\usepackage{newpxtext}
\usepackage[euler-digits,euler-hat-accent]{eulervm}


\usetheme{FJHSlimNoFoot169}
\setlength{\parskip}{2ex}
\setlength{\arraycolsep}{0.5ex}
\setbeamersize{text margin top=10mm}

\DeclareMathOperator{\sol}{SOL}
\DeclareMathOperator{\app}{APP}
\DeclareMathOperator{\alg}{ALG}
\DeclareMathOperator{\ACQ}{ACQ}
\DeclareMathOperator{\ERR}{ERR}
\DeclareMathOperator{\COST}{COST}
\DeclareMathOperator{\COMP}{COMP}
\newcommand{\dataN}{\bigl(\hf(\vk_i)\bigr)_{i=1}^n}
\newcommand{\dataNj}{\bigl(\hf(\vk_i)\bigr)_{i=1}^{n_j}}
\newcommand{\dataNjd}{\bigl(\hf(\vk_i)\bigr)_{i=1}^{n_{j^\dagger}}}
\newcommand{\ERRN}{\ERR\bigl(\dataN,n\bigr)}

\newcommand{\Sapp}{S_{\textup{app}}}
\newcommand{\LambdaStd}{\Lambda^{\textup{std}}}
\newcommand{\LambdaSer}{\Lambda^{\textup{ser}}}
\newcommand{\LambdaAll}{\Lambda^{\textup{all}}}
\newcommand{\oton}{1\!:\!n}
\newcommand{\talert}[1]{\alert{\text{#1}}}
\DeclareMathOperator{\init}{init}
\DeclareMathOperator{\GP}{\cg\cp}
\newcommand{\MLE}{\textup{EB}}
\newcommand{\mCtheta}{{\mathsf{C}_{\vtheta}}}
\newcommand{\mCInv}{\mathsf{C}^{-1}}

%\DeclareMathOperator{\app}{app}

\providecommand{\HickernellFJ}{H.\xspace}


\iffalse

Challenges in Developing Great MCQMC Software
Fred J. Hickernell, Illinois Institute of Technology, Chicago, IL USA

The process of translating new Monte Carlo and Quasi-Monte Carlo (MCQMC) algorithms into software libraries faces several challenges.  Great software should be easy to use with reasonable default options for the novice and advanced features for the developer or experienced user.  The library architecture must allow for growth.  Ensuring connectivity with other software libraries will facilitate a larger user base of the MCQMC library.  Coding algorithms the right way may significantly improve their runtime or portability.  Since development team members will come and go, the shared wisdom of the development team must be documented and transmitted to succeeding generations.  Software developers must keep abreast of the newest computing environments to ensure peak performance.  This talk will highlight some of these challenges and ways to address them.


\fi

\renewcommand{\OffTitleLength}{-10ex}
\setlength{\FJHThankYouMessageOffset}{-8ex}
\title{Challenges in Developing Great MCQMC Software}
\author[]{Fred J. Hickernell}
\institute{Department of Applied Mathematics \&
	Center for Interdisciplinary Scientific Computation \\  Illinois Institute of Technology \quad
	\href{mailto:hickernell@iit.edu}{\url{hickernell@iit.edu}} \quad
	\href{http://mypages.iit.edu/~hickernell}{\url{mypages.iit.edu/~hickernell}}}

\thanksnote{with input from Mark Klinchin, the GAIL an QMCPy teams, and  friends \\
	partially supported by SigOpt, an Intel company \\[2ex]
	Please join us Tuesday, July 19, at 12:45 PM for lunch at Bella Casa \\ to discuss the future of MCQMC software \\[2ex]
	Thanks to the organizers as we meet again in person\\
	Slides at  \href{??}{\nolinkurl{speakerdeck.com/fjhickernell/quasi-monte-carlo-software}}
}
\event{MCQMC 2022, Linz}
\date[]{July 18, 2022}

\input FJHDef.tex


\newlength{\figwidth}
\setlength{\figwidth}{0.25\textwidth}

\newlength{\figwidthSmall}
\setlength{\figwidthSmall}{0.2\textwidth}

\newcommand{\financePict}{\href{http://i2.cdn.turner.com/money/dam/assets/130611131918-chicago-board-options-exchange-1024x576.jpg}{\includegraphics[width
		= 3cm]{ProgramsImages/130611131918-chicago-board-options-exchange-1024x576.jpg}}}
	
	\newcommand{\scoop}[1]{\parbox{#1}{\includegraphics[width=#1]{IceCreamScoop.eps}}\xspace}
	\newcommand{\smallscoop}{\scoop{1cm}}
	\newcommand{\medscoop}{\scoop{1.8cm}}
	\newcommand{\largescoop}{\scoop{3cm}}
	\newcommand{\ICcone}[1]{\parbox{#1}{\includegraphics[width=#1,angle=270]{MediumWaffleCone.eps}}\xspace}
	\newcommand{\medcone}{\ICcone{1.2cm}}
	\newcommand{\largercone}{\parbox{2.2cm}{\vspace*{-0.2cm}\includegraphics[width=1cm,angle=270]{MediumWaffleCone.eps}}\xspace}
	\newcommand{\largecone}{\ICcone{1.8cm}}
	\newcommand{\smallcone}{\parbox{1.1cm}{\includegraphics[width=0.5cm,angle=270]{MediumWaffleCone.eps}}\xspace}

	

\newcommand{\northeaststuff}[3]{
	\begin{tikzpicture}[remember picture, overlay]
	\node [shift={(-#1 cm,-#2 cm)}]  at (current page.north east){#3};
	\end{tikzpicture}}


\begin{document}
	\tikzstyle{every picture}+=[remember picture]
	\everymath{\displaystyle}

\frame{\titlepage}


\section{Introduction}

\begin{frame}{My Definition of Great MCQMC Software}
	
	%\vspace{-5ex}
	\begin{itemize}
		\item \alert{Correct}, e.g., not omit the zeroth point in a low discrepancy sequence \parencite{Owe22a, scipySobol2020a}
	
		\item  \alert{Complete}---contain the components or  easily access components in other libraries to solve real, complex problems
		
		\item  \alert{Accessible}---tutorials, demos, discussion forums, etc.\ for (new) users; written in a language that potential users speak; provide a consistent user interface
		
		\item \alert{Efficient}---in terms of compute time and memory
				
		\item \alert{Current}---include the latest and best algorithms

		\item  \alert{Sustainable}---have a sufficient user base and developer community for updates and maintenance
		
		\item \alert{Scalable}---take advantage of advanced computer architectures for speed and to solve large problems
		
	\end{itemize}
\end{frame}


\begin{frame}[allowframebreaks]{Selective History of Publicly Available MC\textbf{QMC} Software}
\vspace{-2ex}
	\begin{itemize}
		\item \textcite{LEc2017a} provides a history of \alert{random number generation}
		\item Notorious \alert{RANDU} from the 1960s \parencite{RANDU}, which fails the spectral test
		\item \alert{VEGAS} \parencite{Lep78a, Lep21a} importance sampling Monte Carlo algorithm favored by physicists
		\item \alert{ACM} low discrepancy point generators \parencite{BraFox88,BraFoxNie92,HonHic00a}
		\item \alert{FinDer} \parencite{PasTra95,FinDer} and \alert{BRODA} \parencite{BRODA20a} targeting quantitative finance
		\item Korobov cubature and scrambled Sobol' generators in \alert{NAG} \parencite{NAG27} for decades		
		\item Scrambled Sobol' and Halton generators in \alert{MATLAB} \parencite{MAT9.13} since 2008, and fixed a few years later
		\item \alert{BUGS} \parencite{BUGSBook, BUGSweb} and \alert{Stan} \parencite{STAN} for Markov Chain Monte Carlo (MCMC)
		\item \alert{LatMRG} \parencite{LEcCou97}, \alert{RNGStreams} \parencite{LEcEtal02},  \alert{SSJ} \parencite{LEc2002a,SSJ}, and \alert{LatNetBuilder} \parencite{LatNet}
		\item Low discrepancy generators \parencite{FriKel02,FriKelweb}, \alert{SamplePack} \parencite{SamplePack}, \textcite{GruWeb},  and \alert{MatBuilder} \parencite{paulin2022}
		\item \alert{Fast CBC}, \alert{Magic Point Shop}, \alert{QMC4PDE} and other code since 2004  at \textcite{NuyWeb}
		\item Multi-level software \parencite{GilesSoft,GilesQSoft} 
		\item Data-driven error bounds and stopping criteria in \alert{GAIL} \parencite{ChoEtal21a} and \alert{QMCPy} \parencite{QMCPy2020a, ChoEtal22a} 
		\item Uncertainty quantification libraries \alert{Dakota} \parencite{DakotaUsersManual}, \alert{UQTk} \parencite{DebEtal04,UQTk}, and \alert{MUQ} \parencite{MUQ} have some basic level low discrepancy sampling
	    \item QMC framework in Julia since 2019 \parencite{QMCJulia} by Robbe and others
		\item Scrambled Sobol' in \alert{SciPy} \parencite{virtanen2020scipy} and \alert{PyTorch} \parencite{paszke2019pytorch} since several years ago
		\item \alert{TensorFlow} QMC framework \parencite{tfqfQMC2021a} since a year or so ago
	\end{itemize}

In summary, we have stand-alone algorithms, small libraries, and pieces of larger libraries
\end{frame}

\begin{frame}{Why Do We Need Software?}
	\vspace{-5ex}
	\begin{itemize}
		\item \alert{Theory} 
		\begin{itemize}
			\item Explains and justifies \alert{algorithms}
			\item Assumptions$\implies$success should be viewed as failure$\implies$why
		\end{itemize}
		\item \alert{Software} 
		\begin{itemize}
			\item Makes \alert{algorithms} practical
			\item Solves (new) \alert{applications}
			\item Eliminates \alert{do-it-yourself}
		\end{itemize}
		\item \alert{Applications} 
  		\begin{itemize}
			\item Make \alert{societal} impact
			\item Inspire new \alert{theory}
		\end{itemize}

	\end{itemize}

\uncover<2->{The United States Department of Energy is investing in studying how to develop great scientific software \parencite{Her19a,ASCR-SSSDU}}
\end{frame}

\section{Structure}

\begin{frame}{Software Library Architecture}
	\vspace{-7ex}

\[
\mu :=  \int_{\ct} g(\vt) \, \lambda(\vt) \, \dif \vt = \cdots = 	
\underbrace{\Ex[f(\vX)]}_{\text{expectation}} = \underbrace{\int_{[0,1]^d}  f(\vx) \, \dif \vx}_{\text{integration}} \approx  \frac 1{n} \sum_{i=1}^{n} f(\vX_i) =: \hmu_{n}
\]

\vspace{-3ex}

\begin{description}
	%\setlength{\itemsep}{0.2cm}
	
	\item[LD Generator] producing $\{\vX_1, \vX_2, \dots \}$ that mimics the distribution with PDF $\varrho$, e.g., uniform
	
	\item[True Measure] that defines the original integral, e.g., Lebesgue; embodies the transformation $\vt = \Psi(\vx)$
	
	\item[Integrand] $g$, which defines the original integral, plus the transformed version, $f$, to fit the LD generator 
	
	\item[Stopping Criterion] based on a data-driven error bound, which determines how large $n$ should be to ensure that $\abs{\mu - \hmu_n} \le \varepsilon$
\end{description}

\vspace{-1ex}
\uncover<2->{May we agree on a \alert{common definition} for these these objects like we have for floating point numbers or basic mathematical functions?}
\end{frame}

\begin{frame}{Choice of Language, Library, or Environment}
	\begin{itemize}
    \item Language/library choices of the users are driven by familiarity and speed

    \item Several favorites aligned with various communities, but no one dominates all

    \item Growth of multi-processor environments 
    \begin{itemize}
    	\item Provides opportunity
    	\item Complicates software development; where does parallelism really help?
    \end{itemize}

    \item<2-> Recommendations
    \begin{itemize}
    \item Contribute your new idea as a demo in your favorite library
    \item If your contribution is substantial, add as a feature to other libraries
    \item Move basic features of your library to larger library
    \item Write wrappers or demos to connect other libraries to your library
    
    \end{itemize}
 \end{itemize}
\end{frame}

\begin{frame}{Ex.\ Connecting QMCPy with FEniCS/Dolfin \parencite{}}
\vspace{-4ex}
	\begin{itemize}
		\item Want to demonstrate how our QMC software could work with a popular DE solver for DEs with random coefficients
		\item Partial success \only<3->{\alert{Why?}}
	\end{itemize}
 	\vspace{-2ex}
 \begin{minipage}{0.48\textwidth}
 	\only<1-2>{\begin{gather*}
 		-\frac{\dif }{\dif x}\biggl(a(x,\vW) \frac{\dif}{\dif x} u(x,\vW) \biggr) = 200( 2x - 1)\\
 		 0 \leq x \leq 1, \qquad
 		u(0) = u(1) = 0 \\
 		a(x,\vW) = 1 + 0.6 \sum_{k=1}^d \frac{W_k T_{k}(2x-1)}{k^2} \\
 		 \vW \sim \cu[-1,1]^d \\
 		 \Ex[u(0.25,\vW)] = \only<1>{\, ?}\only<2>{ -1.7462}
 	\end{gather*}}
  \only<3->{
\begin{itemize}
\item Even with FEniCS/Dolfin's extensive documentation we had \alert{difficulties}
\begin{itemize}
    \item Learning how to solve a simple ODE, so reverted to an older version
    \item Learning how to change the random instance of coefficient without tiggering the JIT compile
    \item Do not yet know how to express the random coefficient in terms of the covariance kernel
\end{itemize}
\end{itemize}
  }
 \end{minipage}%
\hfill
 \begin{minipage}{0.48\textwidth}
 	\only<1-2>{\centering
 	\includegraphics<1>[width=\textwidth]{ProgramsImages/axwtrue0.6.eps}
 	\includegraphics<2->[width=\textwidth]{ProgramsImages/timing-4.eps}}%
  \only<3->{
\begin{itemize}
\item But
\begin{itemize}
    \item It does work
    \item If we can overcome the challenges, will post a blog and demo at \textcite{QMCBlog}
\end{itemize}
\end{itemize}

  }
\end{minipage}

\end{frame}

\section{Support}
\begin{frame}{Encouraging Users Requires}
	\begin{itemize}
		\item A repository where
		\begin{itemize}
		\item  Your library can be easily downloaded
		\item Issues or queries can be posted
		\item There are updates
		\end{itemize}
		\item (Elementary) \alert{demos} on how to use the software and highlighting  its advantages
		\item Demos \alert{connecting}  your library with others
		\item Recommending your students and postdocs to \alert{use} your library
		\item \alert{Welcoming} instructions on how to contribute an algorithm or a demo
	\end{itemize}
		
\end{frame}


\begin{frame}{Building Developer Team}
	
\begin{itemize}
	
\item Many senior researchers \alert{don't code}, and rely on transient team members (students, postdocs)
\begin{itemize}
	\item Encourage a sense of ownership that lasts beyond the time your team members are part of your lab
	\item Encourage 
\end{itemize}

\item Recognize good software as \alert{valuable} scholarly output
	
\item Recognize \alert{research software engineer} as a valued vocation; look at \alert{Research Software Engineers International} 	\url{https://researchsoftware.org}, \emph{Research Software Engineers are people who combine professional software expertise with an understanding of research. They go by various job titles but the term Research Software Engineer (RSE) is fast gaining international recognition.}
  \end{itemize}
\end{frame}

\section{Conclusion}

\begin{frame}{Next Steps}
	\includegraphics[width=0.5\textwidth]{ProgramsImages/PizeriaBellaCasa.png}
\end{frame}

\begin{frame}[allowframebreaks]{References}
	\printbibliography
\end{frame}

\end{document}

\begin{frame}{Quasi-Monte Carlo (QMC) uses low discrepancy (LD) sequences}
	\vspace{-3ex}
	\begin{tabular}{>{\centering}p{0.47\textwidth}@{\quad}>{\centering}p{0.47\textwidth}}
		%Independent \& Identically Distributed (IID) &
		%Low Discrepancy (LD) \tabularnewline
		\includegraphics[height=5cm]{ProgramsImages/IIDPoints.eps} &
		\includegraphics[height=5cm]{ProgramsImages/SSobolPoints.eps}
		\tabularnewline
		$\vT_i$ are random &
		$\vX_i$ may be deterministic \alert{or} random 
		\tabularnewline
		$\vT_1, \vT_2 \cdots \alert{\IIDsim} F$ &
		$\vX_1, \vX_2 \cdots \alert{\LDsim} F$ 
		\tabularnewline
		$\vT_i$ do not know about one another &
		$\{\vX_i\}_{i=1}^n$ represent $F$ well
		\tabularnewline
		$F_{n}(\vt_1, \ldots, \vt_n) = F(\vt_1) \cdots F(\vt_n)$ &
		$F_{\{\vX_i\}_{i=1}^n}(\vx) \approx F(\vx)$
	\end{tabular}
\end{frame}

\begin{frame}{You have heard about QMC.  How do you try it?}
	
	\vspace{-5ex}
\begin{itemize}
\setlength{\itemsep}{0cm}
    \item<1-> \emph{QMC will give you 100 times the accuracy in the same amount of time as simple MC} \\
     \uncover<2->{\alert{Often}}
    \item<1-> \emph{Just replace your IID random points with low discrepancy (LD) points}\\
    \uncover<2->{\alert{Sometimes}}
    
    \vspace{4ex}
    
    \item<3-> \emph{Where can I get accurate, efficient, easy to use QMC software to try for my problem?}\\
     \item<3-> \emph{How can I make my great QMC software available for others?}\\
    \uncover<4->{\alert{Let's try to help}}
    
\end{itemize}
\end{frame}

\begin{frame}{Quasi-Monte Carlo (QMC) uses low discrepancy (LD) sequences}
\vspace{-3ex}
\begin{tabular}{>{\centering}p{0.47\textwidth}@{\quad}>{\centering}p{0.47\textwidth}}
%Independent \& Identically Distributed (IID) &
%Low Discrepancy (LD) \tabularnewline
\includegraphics[height=5cm]{ProgramsImages/IIDPoints.eps} &
\includegraphics[height=5cm]{ProgramsImages/SSobolPoints.eps}
\tabularnewline
$\vT_i$ are random &
$\vX_i$ may be deterministic \alert{or} random 
\tabularnewline
$\vT_1, \vT_2 \cdots \alert{\IIDsim} F$ &
$\vX_1, \vX_2 \cdots \alert{\LDsim} F$ 
\tabularnewline
$\vT_i$ do not know about one another &
$\{\vX_i\}_{i=1}^n$ represent $F$ well
\tabularnewline
$F_{n}(\vt_1, \ldots, \vt_n) = F(\vt_1) \cdots F(\vt_n)$ &
$F_{\{\vX_i\}_{i=1}^n}(\vx) \approx F(\vx)$
\end{tabular}
\end{frame}

\begin{frame}{Where is the software?  (apologies to those I missed)}
			\vspace{-3ex}
	
	\renewcommand{\arraystretch}{1.15}
	\begin{tabular}{>{\centering}m{0.47\textwidth}@{\qquad}>{\centering}m{0.47\textwidth}}
		\alert{LD Sequence Generators} & \alert{Multi-Level, Stopping Criteria, Applications}
		\tabularnewline \toprule
		\uncover<1>{\href{https://www.mathworks.com}{\alert{MATLAB Statistics Toolbox}}---\newline Sobol' and Halton} &
		\href{https://people.maths.ox.ac.uk/gilesm/mlmc/}{\alert{Mike Giles}}---Multi-Level (Quasi-)Monte Carlo  \uncover<1>{in C++, MATLAB, Python, and R}
		\tabularnewline
		\href{https://cran.r-project.org/web/packages/qrng/qrng.pdf}{\alert{Marius Hofert \& Christiane Lemieux }}---\texttt{qrng} \uncover<1>{R package,} Sobol' and Halton &
	  \href{http://gailgithub.github.io/GAIL_Dev/}{\alert{Guaranteed Automatic Integration Library (GAIL)}}---Stopping criteria  \uncover<1>{in MATLAB}
		\tabularnewline
		\uncover<1>{\href{http://www.broda.co.uk}{\alert{BRODA, Sergei Kucherenko}}---Sobol' in C, MATLAB, and Excel}& 
		\uncover<1>{\href{https://www.uqlab.com}{\alert{UQLab}}---Framework for Uncertainty Quantification in MATLAB}
		\tabularnewline
		\href{http://statweb.stanford.edu/~owen/code/}{\alert{Art Owen}}---randomized Halton\uncover<1>{ in R}&
		\multirow{3}{0.47\textwidth}{\uncover<1>{\centering \href{http://www.openturns.org}{\alert{OpenTURNS}---An Open source initiative for the Treatment of Uncertainties, Risks 'N Statistics in Python}}}
		\tabularnewline
		\uncover<1>{\href{https://github.com/PieterjanRobbe/QMC.jl}{\alert{Pieterjan Robbe}---LD sequences in Julia}}
		\tabularnewline
		\href{https://pytorch.org/}{\alert{PyTorch}---Sobol' \uncover<1>{in Python}}
		\tabularnewline
		\multicolumn{2}{>{\centering}m{0.96\textwidth}}{\href{http://simul.iro.umontreal.ca}{\alert{Pierre L'Ecuyer}---Lattice Builder \uncover<1>{and  Stochastic Simulation in C/C++ and Java}}}
		\tabularnewline
		\multicolumn{2}{>{\centering}m{0.96\textwidth}}{\href{https://people.cs.kuleuven.be/~dirk.nuyens/}{\alert{Dirk Nuyens}}---Magic Point Shop \uncover<1>{and QMC4PDE in MATLAB, Python, and C++}}
\tabularnewline
		\multicolumn{2}{>{\centering}m{0.96\textwidth}}{\uncover<1>{\href{http://people.sc.fsu.edu/~jburkardt/}{\alert{John Burkhardt}}---variety in C++, Fortran, MATLAB, \& Python}}
\tabularnewline
		\multicolumn{2}{>{\centering}m{0.96\textwidth}}{\href{https://qmcsoftware.github.io/QMCSoftware/}{\alert{QMCPy}}---Python package \alert<2->{incorporating and connecting} the work of different groups}
\tabularnewline
	\end{tabular}

\renewcommand{\arraystretch}{1}
    
\end{frame}


\begin{frame}{Why LD is better than IID}
	\vspace{-4ex}
	\begin{description}
		\setlength{\itemsep}{0.5cm}
		\item[Integration/Expectation]  Arising in finance, uncertainty quantification, Bayesian inference, \ldots
		\begin{equation*}
		\mu := \Ex[f(\vX)] = \int_\cx f(\vx) \, \varrho(\vx) \, \dif \vx \approx \frac 1n \sum_{i=1}^n f(\vX_i) =:  \hmu_n
		\end{equation*}
		LD points give faster convergence than IID
		
		\item[Design of Computer Experiments] LD can be more space filling (even than Latin hypercube sampling) for use in constructing surrogate models and uncertainty quantification

		\item[Global Optimization]  LD points are more space filling and find good starting points for local methods
		
		\end{description}
	
    
\end{frame}

\begin{frame}{What software components do we need?}
	
	\vspace{-6ex}
	
	\[
	 \uncover<2->{\mu :=  \int_{\ct} \alert<3>{g(\vt)} \, \alert<2>{\lambda(\vt) \, \dif \vt} = \cdots = } \only<1>{\mu :=} 	\underbrace{\Ex[f(\vX)]}_{\text{expectation}} = \underbrace{\int_\cx \alert<3>{f(\vx)} \, \varrho(\vx) \, \dif \vx}_{\text{integration}} \approx  \frac 1{\alert<4>{n}} \sum_{i=1}^{\alert<4>{n}} f(\alert<1>{\vX_i}) =: \hmu_{\alert<4>{n}}
	\]
	
		\vspace{-3ex}
	
	\begin{description}[<+->]
				\setlength{\itemsep}{0.5cm}
		
		\item[LD Generator] producing $\{\vX_1, \vX_2, \dots \}$ that mimics the distribution with PDF $\varrho$, e.g., uniform
		
		\item[True Measure] that defines the original integral, e.g., Lebesgue
		
		\item[Integrand] $g$, which defines the original integral, plus the transformed version, $f$, to fit the LD generator
		
		\item[Stopping Criterion] that determines how large $n$ should be to ensure that $\abs{\mu - \hmu_n} \le \varepsilon$
	\end{description}
\end{frame}

\section{QMCPy on Google Colaboratory}
\begin{frame}{QMCPy in a Jupyter Notebooks on Google Colaboratory}
	\vspace{-4ex}
	
	\alert{Prerequisites}---No Python, Jupyter, or QMC knowledge assumed
	
	\vspace{-1ex}
	
	\alert{Goals}
	
		\vspace{-3ex}
		\begin{itemize}
		\item Show you how QMC software works
		\item Interest you in using/contributing
	\end{itemize}
	
	\vspace{-1ex}

	\alert{Directions}
	
	\vspace{-3ex}
	\begin{itemize}
		\item Point your browser to \href{https://tinyurl.com/QMCPyTutorial}{\nolinkurl{https://tinyurl.com/QMCPyTutorial}}
		\item Open the file in Google Colaboratory (may need to push a button at the top of your browser)
		\item Make a copy of this file onto your own Google drive account \\
		\texttt{File} $\rightarrow$ \texttt{Save a copy in Drive}
	\end{itemize}



\alert{Pause for questions}
\end{frame}

\section{Why collaborate?}

\begin{frame}
	\frametitle{Acknowledgments for QMCPy}
	
	\begin{itemize}
		\item Coded almost entirely by Aleksei Sorokin (BS/MS '21)
		
		\item Relies on code from several groups (see above)
		
		\item Documentation and testing by Sou-Cheng Choi and Jagadeeswaran R.
		
		\item Funded and encouraged by Mike McCourt at SigOpt
	\end{itemize}
\end{frame}


\begin{frame}
	{The Guaranteed Automatic Integration Library (GAIL) and QMCPy teams}
	
	\vspace{-2ex}
	\includegraphics[angle = 180, origin = c, width = 0.32\textwidth]{ProgramsImages/GAIL2014RE.jpeg} \
	\includegraphics[width = 0.32\textwidth]{ProgramsImages/GAILatSIAM2018Hi.jpeg} \ 
	\includegraphics[width = 0.32\textwidth]{ProgramsImages/GAILatChinatown2018.jpg}
	
	\vspace{-4ex}

	{\small 
		\hspace{-4ex}\begin{tabular}{p{0.545\textwidth}p{0.44\textwidth}}
		
		\begin{itemize}
			\setlength{\itemsep}{0ex}
	
			\item Sou-Cheng Choi (Chief Data Scientist, Kamakura)
			
			\item Yuhan Ding (IIT PhD '15, Lecturer, IIT)
			
			\item Lan Jiang  (IIT PhD '16, Compass)
			
			\item Llu\'is Antoni Jim\'enez Rugama (IIT PhD '17, UBS)
			
			\item Mike McCourt (IIT BS '07, Cornell PhD '12, \\ Head of Research, SigOpt)
			
			\item Jagadeeswaran Rathinavel (IIT PhD '19, Wi-Tronix)
			
			
			
			
		\end{itemize}
		
		&
		
		\begin{itemize}
			
			\setlength{\itemsep}{0ex}
			
			\item Aleksei Sorokin (IIT BS \& MAS '21 exp.)
			
			\item  Xin Tong (IIT MS, UIC PhD '20 exp.)
			
			\item Kan Zhang (IIT PhD '20 exp.)
			
			\item Yizhi Zhang (IIT PhD '18, Jamran Int'l)
			
			\item Xuan Zhou (IIT PhD '15, JP Morgan)
			
			\item and others
			
			
		\end{itemize}
	
		
	\end{tabular}
}

\end{frame}

\begin{frame}{The argument for community software}
	
	\vspace{-4ex}
	
	\begin{itemize}
	\item<+-> Our research groups are typically expert at only part of the whole picture:
			\begin{itemize}
			\item LD sequence generators
			
			\item Increasing efficiency, e.g., MLMC, MDM
			
			\item Stopping criteria
			
			\item Use cases
		\end{itemize}
	
		A community library lets us all take advantage of the best

	\item<+-> Provides a consistent interface for pieces from different places
	
	\item<+-> Enables reproducible computational research
	
	\item<+-> Tedious stuff only done once

		\item<+-> Many eyes find and correct errors
		\begin{itemize}
			\item MATLAB's Sobol' generator's  scrambling corrected in MATLAB 2017a after Tony Jim\'enez Rugama noticed the problem
			
			\item PyTorch's Sobol' generator found to be wrong unless double precision is proactively specified; also missing the first point; reported at \url{https://github.com/pytorch/pytorch/issues/32047}
		\end{itemize}
	

\end{itemize}
\end{frame}

\begin{frame}{How you can contribute}
	
		\vspace{-3ex}
	Try out  QMCPy and then
	
			\vspace{-1ex}
	
	
	\begin{description}
		\item[Easy] Submit your bugs and feature requests as issues to \url{https://github.com/QMCSoftware/QMCSoftware/issues}
		
		\item[Moderately Difficult]  Ask your students or collaborators to try QMCPy themselves and submit their bugs and feature requests\\[1ex]
		
		Email us your blog post to add to \href{https://qmcpy.wordpress.com/}{\nolinkurl{qmcpy.wordpress.com/}}
		
		\item[Heroic] Add a feature or use case and make a pull request at \url{https://github.com/QMCSoftware/QMCSoftware/pulls} \\ so that we can included it in our next release
	\end{description}

		\vspace{-1ex}
	
	Questions or suggestions?  Email us at \href{mailto:qmc-software@googlegroups.com}{\nolinkurl{qmc-software@googlegroups.com}}
	
			\vspace{-1ex}
	
	\uncover<2>{If you are wedded to another language, think of designing your software so that others can add to it easily.}

\end{frame}

\finalthanksnote{These slides are  available at \\  \href{https://speakerdeck.com/fjhickernell/quasi-monte-carlo-software}{\nolinkurl{speakerdeck.com/fjhickernell/quasi-monte-carlo-software}}\\
Google Colaboratory notebook at \href{https://tinyurl.com/QMCPyTutorial}{\nolinkurl{tinyurl.com/QMCPyTutorial}}\\
Blog at \href{https://qmcpy.wordpress.com/}{\nolinkurl{qmcpy.wordpress.com/}}}


\thankyouframe


\end{document}





